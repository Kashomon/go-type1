\documentclass{article} %% baduk.tex :: mito / nyamcoder %%

% \usepackage{fontspec}
% \font\kg="HY백송B" at 14pt		% ttf; needs HWP installed -- or choose similar a font
% \font\kf="HY백송B" at 10pt		%           -- dito --
% \font\kb="은 자모 바탕" at 10pt		% ttf; Un font collection, download from here: http://kldp.net/projects/unfonts/download
						% after unzipping and copying to your font folder, in Linux don't forget to run ``fc-cache -fv''

\usepackage{igo}				% not included in TL -- needs extra installation
\igofontsize{10}

\begin{document}
\centering
%\shortstack{
% \kg 흑으로 백을 잡는 문제\\[20pt]
%\hspace{.28mm}
\flushleft\hspace*{1.97cm}
\begin{minipage}{3.5cm}
\black{a7,a8,b2,b8,c2,c8,d2,d3,d4,d5,d6,d7,d8}
\white{a3,a6,b3,b5,c3,c4}
\showgoban[a1,f10]
\end{minipage}
\begin{minipage}{3.5cm}
% \kb 집을 낼수 없도록 하는 급소를 찾아서 두는 문제입니다.
\end{minipage}
%}
\\[1em]
%\shortstack{
\centering
% \kf 정답\\[.3cm]
\black[1]{c5}
\showgoban[a1,f10]\cleargobansymbols\quad
\white[2]{a4}
\showgoban[a1,f10]\cleargobansymbols\quad
\black[3]{b6}\white[\igocross]{a6}
\showgoban[a1,f10]%\cleargobansymbols
%}
\end{document}
